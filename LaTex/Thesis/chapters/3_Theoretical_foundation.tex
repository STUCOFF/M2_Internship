\chapter{Theoretical Foundations for the Solution}
\textit{This is a generic title. Replace it with an actual title that describes the context of the work. \\
Describe in abstract (theoretical) terms how the proposed approach can be implemented and how to solve related sub problems.  Use the state of the art as an analysis tool. }\\
\\
\\
\section{Computational Tomography}
\begin{itemize}
	\item brief descrition of basics of CT reconstrution, Define Radon Transform
	\item show an image and output sinogram explain...
\end{itemize}

\section{Syncrothron Radiation Nano-CT reconstruction}
\begin{itemize}
	\item brief mention of phase retrieval, Max thesis
	\item Wave monochromatique incoherent
	\item 1 formula refractive index
	\item Attenuation: wave/shift
	
	\item Description of tomographic reconstruction (focus on, apply CS)
	
\end{itemize}

\section{Compressed sensing}
\begin{itemize}
	\item Candes more details that state of the art
	\item CS in nano CT: iterative SB reconstruction proposed (Juan ART-TV SB, Pesquet Scalable splitting)
\end{itemize}