\chapter{State-of-the-Art}
	intro about CT and importance for osteoporosis diagnosis + use of SR + low dose problem CS (\cite{[24], [25], [26]})

\section{Dose reduction in SR Micro-CT}
Multiple CS algorithm were developed for Micro-CT allowing to generate different spacial resolutions. Alternative methods then FBP necessary to recover missing projections. Iterative algorithms are used.
	\subsection{No SR}
		SART-L1 \cite{[11],[13]} ASD-POCS TV \cite{[9]}
	\subsection{CS on SR micro-CT}
		multiple iterative methods using CGTV (\cite{[12]}) ART with multiple denoising (TV \cite{[3]}; L1 minimisation \cite{[18]}; Discrete packet shrinkage \cite{[2]}) SART (\cite{[1]} with TV \cite{[5]}) OS-SART \cite{[6]}) EST \cite{[15], [16]}  PCCT \cite{[8]}
		 define resolution for each solution (maybe more details?)

\section{SR Nano-CT}
	Nano-CT general ref: \cite{[23]} (I can have other references but are mostly about the hardware side, new materials and acquisition methodology, or image post-processing without having used low dose)\\
	less CS reconstruction experimented	\\
	Low dose nano OS-SART L1 norm TV \cite{[10]}

\section{Wrap-up}
	A lot of research these past few years of CSCT going toward a improvement of spacial resolution and dose reduction. Yet not so much has been done on Nano scale. In the context of osteoporosis nano scale is mandatory for a accurate diagnosis. Present our objective.
