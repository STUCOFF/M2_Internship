\chapter{Proposed approach}
We propose to use iterative reconstruction with the algorithm described by \cite{goldstein2009split}. Split Bregman algorithm gives a solution to an L1-L2 constrained problem. We will here describe the bregman iteration and it's application to L2 minimisation reconstructions which will be used in our algorithm.

\section{Split Bregman iterative reconstruction}
    \subsection{Split Bregman iteration}
    Using Split Bregman we wish to solve the constrained reconstruction optimization problem described in the section \ref{}:
    \begin{equation}
        \min\limits_{u}||\nabla_u||_1 \mbox{ such that } Fu=f
        \label{Eq:ConstPblm}
    \end{equation}
 Such constrained problem problems are difficult to solve directly. For this reason we need to define a new unconstrained problem. Luckily it is possible to approximate (\ref{Eq:ConstPblm}) as:
    \begin{equation}
        \min_{u}||\nabla_u||_1 + \frac{\lambda}{2}||Fu-f||_2^2
    \end{equation}
    
    The Bregman iteration allows us to reduce \ref{Eq:ConstPblm} in even shorter unconstrained problems using Bregman distances. These constrained problem can be resolved iteratively as follows:
    \begin{equation}
        \begin{aligned}
            u^{k+1} &= \min_{u} ||\nabla_u||_1 + \frac{\lambda}{2}||Fu - f^k||_2^2 \\
           f^{k+1} &= f^k + f - Fu^k
        \end{aligned}
    \end{equation}
    \subsection{L1 regularization problem}
    Our compressed sensing reconstruction method is based on L1 regulation. A more faithful reconstruction problem must be formulated and we will see how to solve it iteratively with split Bregman.\\
    The idea is to "de-couple" the L1 and L2 parts of our original problem. We wish to minimize the Total Variation $\nabla_u$ of the image and a weight function $H()$. We write the problem as follows:
    \begin{equation}
        \label{Eq:L1Const}
        \min_{u,d}||d||_1 + H(u) \mbox{such that} d = \nabla_u
    \end{equation}
    Which can be computed iteratively using Split Bregman iteration as:
    \begin{equation}
        \begin{aligned}
            (u^{k+1},d^{k+1}) &= \min_{u,d} ||d||_1 + H(u) +     \frac{\lambda}{2}||d - \nabla_u - b^k||^2_2\\
            b^{k+1} &= b^k + \nabla_{u^{k+1}} - d^{k+1}
    \end{aligned}
    \end{equation}
\section{SB-TV-2D reconstruction}
    isotropic TV denoising pbl: 
    \begin{equation}
        \min_{u} ||\nabla_u||_1 \mbox{ such that } ||Fu - f||_2^2 < \sigma^2
    \end{equation}
    
    where $\nabla_u = (\nabla_x,\nabla_y)u$, $f$ represents the projection space, $F$ the projection operator, $u$ the image domain and $\sigma$ represents the variance of the signal noise.
    \begin{equation}
        \begin{aligned}
            u^{k+1} &= \min_{u}||\nabla_u||_1 + \ \frac{\lambda}{2}||Fu - f^k||_2^2\\
            f^{k+1} &= f^k + f - F_u^k
        \end{aligned}
    \end{equation}

	We fall here into an unconstrained problem which is not steight forwardly solved. In order to get a constrained problem we will insert a variable d such that $d = \nabla_u$.\\
	We can now use the Split Bregman iteration in order to solve our new problem: 
     \begin{equation}
        \begin{aligned}
            u^{k+1} &= \min_{u,d}||d||_1 + \ \frac{\lambda}{2}||Fu - f^k||_2^2 \mbox{ such that } d = \nabla_u\\
            f^{k+1} &= f^k + f - F_u^k
        \end{aligned}
    \end{equation}
	And get to a solution where L1 and L2 elements of our original problem are spleted into two equations:
    \begin{equation}
        \begin{aligned}
            u^{k+1} &= \min_{u} \frac{\mu}{2}||Fu - f^k||_2^2 + \frac{\lambda}{2} ||d^k - \nabla_u - b^k||_2^2\\
            d^{k+1} &= \min_{u} ||d||_1 + \frac{\lambda}{2}||d - \nabla_u-b^k||_2^2\\
            b^{k+1} &= b^k +\nabla_u^{k+1} - d^{k+1}\\
            f^{k+1} &= f^k + f - F_u^k
        \end{aligned}
    \end{equation}
    Now is left to sole the minimization on the $u^{k+1}$ and $d^{k+1}$ operations. Both operation are differentiable. We are hence get the minimum using the derivative.\\
\textbf{Solution for u}\\
\begin{equation}
	\begin{aligned}
		u^{k+1} = 
	\end{aligned}
\end{equation}

\textbf{Solution for d}\\
    
    
    
\section{SB-TV-3D}
    pbl: $\alpha ||(\nabla_x,\nabla_y, \nabla_z)u||_1 \mbox{such that} ||Fu - f||_2^2 < \delta^2$
    
    \begin{equation}
        \begin{aligned}
            u =\\
            d =
        \end{aligned}
    \end{equation}